% !TEX TS-program = pdflatex
% !TEX encoding = UTF-8 Unicode

% This is a simple template for a LaTeX document using the "article" class.
% See "book", "report", "letter" for other types of document.

\documentclass[11pt]{article} % use larger type; default would be 10pt

\usepackage[utf8]{inputenc} % set input encoding (not needed with XeLaTeX)

%%% Examples of Article customizations
% These packages are optional, depending whether you want the features they provide.
% See the LaTeX Companion or other references for full information.

%%% PAGE DIMENSIONS
\usepackage{geometry} % to change the page dimensions
\geometry{a4paper} % or letterpaper (US) or a5paper or....
% \geometry{margin=2in} % for example, change the margins to 2 inches all round
% \geometry{landscape} % set up the page for landscape
%   read geometry.pdf for detailed page layout information

\usepackage{graphicx} % support the \includegraphics command and options

% \usepackage[parfill]{parskip} % Activate to begin paragraphs with an empty line rather than an indent

%%% PACKAGES
\usepackage{booktabs} % for much better looking tables
\usepackage{array} % for better arrays (eg matrices) in maths
\usepackage{paralist} % very flexible & customisable lists (eg. enumerate/itemize, etc.)
\usepackage{verbatim} % adds environment for commenting out blocks of text & for better verbatim
\usepackage{subfig} % make it possible to include more than one captioned figure/table in a single float
\usepackage{amsmath} %math package
\usepackage{graphicx} %used to include images
\usepackage{hyperref}
% These packages are all incorporated in the memoir class to one degree or another...

%%% HEADERS & FOOTERS
\usepackage{fancyhdr} % This should be set AFTER setting up the page geometry
\pagestyle{fancy} % options: empty , plain , fancy
\renewcommand{\headrulewidth}{0pt} % customise the layout...
\lhead{}\chead{}\rhead{}
\lfoot{}\cfoot{\thepage}\rfoot{}

%%% SECTION TITLE APPEARANCE
\usepackage{sectsty}
\allsectionsfont{\sffamily\mdseries\upshape} % (See the fntguide.pdf for font help)
% (This matches ConTeXt defaults)

%%% ToC (table of contents) APPEARANCE
\usepackage[nottoc,notlof,notlot]{tocbibind} % Put the bibliography in the ToC
\usepackage[titles,subfigure]{tocloft} % Alter the style of the Table of Contents
\renewcommand{\cftsecfont}{\rmfamily\mdseries\upshape}
\renewcommand{\cftsecpagefont}{\rmfamily\mdseries\upshape} % No bold!

%%% END Article customizations

%%% The "real" document content comes below...

\title{Simulation de Chaîne Correcteur PID - Moteur CC pour la stabilisation des constantes de GA-optimisateur des constantes PID}
\author{Timo Demos-Tacos}
%\date{} % Activate to display a given date or no date (if empty),
         % otherwise the current date is printed 

\begin{document}
\maketitle

\section{Introduction}

Après avoir implimenté un algorithme génétique capable de corriger de manière automatique des coéfficients PID, on s'intèresse à un autre problème avant de l'utiliser pour faire un asservissement réel : Comment choisir certaines constantes utilisées dans GA, afin d'augmenter la vitesse de convergence et assurer la convergence vers un équilibre maximale? Il y a 2 réponses. Le premier est pûrement théorique : Il faut formaliser mathématiquement le processus d'évolution décrit dans GA et éstimer un coéfficient PID. Dans ce cas un éstimateur est un individu avec le meilleur score dans la population numéro n. Sachant que la première population est générée de manière aléatoire, un individu est réprésenté comme une variable aléatoire (Séction 2). Deuxième approche est de simuler la sortie de la chaîne Correcteur PID - Moteur CC pour des systèmes d'ordre différentes et de trouver, de manière émpirique les meuilleurs constantes possibles (Séction 3).


\subsection{A subsection}

\section{Approche théorique}

\section{Simulation}

\subsection{Fonction de Transfert}

Calculons d'abord la fonction de transfert du système qu'on veut asservir. Comme système on choisit le robot qui se déplace grâce à un moteur de courant continue. On a, donc un schèma bloc suivant, qui décrit une partie de la chaîne du système :
\includegraphics[width=16cm, height=10cm]{bloc1}

\subsection{Fonction de transfert du moteur à courant continu (M(p))}

On peut modéliser le moteur CC par une approche classique. On obtient un modèle équivalent en régime dynamique :

\includegraphics[width=12cm, height=8cm]{schem_induit}

Il est facile de démontrer qu'en appliquant le principe fondamental de la dynamique, il est facile d'obtenir la fonction de transfert :


\begin{equation} M(p) = \frac{K_0}{1 + (\tau + \tau e)p + \tau^2 ep^2} \end{equation} 

où \begin{math}  \tau = \frac{RJ}{K^2 + Rf} \end{math}, et \begin{math} \tau e = \frac{L}{R}\end{math}, et \begin{math} K_0 = \frac{K}{K^2 + Rf} \end{math}

Pour plus d'informations, voir \href{http://physiquenetappliquee.free.fr/Modele_dyna_MCC.php}{ici, par exemple}

On les ordres de grandeur de moteurs utilisés(nom du moteur) : 
\begin{equation} R=20\Omega ,  J=25 kg  \cdot m^2 , f=0.67 , L =45mH , K=13 \end{equation}

D'où \begin{math} \tau e = 2123 , \tau = 3535 , K_0 = 352353\end{math}

\subsection{Fonction de transfert du système mécanique (R(p))}

On modélise la courbure du sol par une simple sinusoide \begin{equation}  r(t) = A \cdot \sin (\omega_0 t) \end{equation}

Comme l'angle d'inclinaison est très petite on a : \begin{equation} N = mg , \sin \alpha \approx \tan \alpha \approx \alpha\end{equation} 

La liaison moteur-roue est un pivot idéal.

La force des frottements fluides(projection en x, en supposant la vitesse faible) : \begin{equation} -K_f \cdot v \end{equation}

La force motrice : \begin{equation} F_m = \frac{J}{r} \cdot \frac{d\omega}{dt}\end{equation} 

La force des frottements solides (en supposant système à l'équilibre dynamique, se déplaçant depuis longtemps à une vitesse de consigne) : \begin{equation} -K_s \sigma mg  \end{equation}
On a \begin{math}  \sigma =\frac{A \omega_0 r \epsilon}{2<v>^2} \cdot v \end{math}

Les ordres de grandeurs et la description des constantes utilisées : 

\begin{math} A = 0.001m \end{math} est une amplitude de la courbure,  \begin{math} m=5kg \end{math} la masse de robot, l'angle moyen de surface avec l'horizontal, \begin{math} r=0.3m \end{math} rayon de la roue, \begin{math} J = 25 kg \cdot m^2 \end{math} moment d'inertie de la roue, \begin{math} \epsilon  = 0.01m \end{math} l'épaisseur de la roue, \begin{math} <v>=1 \frac{m}{s} \end{math} la vitesse moyenne, \begin{math} K_f = 0.65 \end{math} la constante des frottements fluides, la constante des frottements solides et \begin{math} K_s = 0.05 \frac{1}{m^2} \end{math} la constante de frottements solide par unité de surface.

On obtient une équation différentielle (PFD) :

\begin{equation} \frac{J}{r} \cdot \frac{d\omega}{dt} - K_f \cdot v - K_s \frac{A \omega_0 r \epsilon}{2<v>^2} \cdot v mg = m \cdot \frac{dv}{dt}\end{equation}

Alors

\begin{equation} R(p) = \frac{\frac{J}{r K_1}p}{\frac{m}{K_1}p + 1} , K_1 = K_f + K_s \cdot \frac{A\omega_0 r \epsilon mg}{2<v>^2}\end{equation}

On \begin{math} K_1 \approx 3243 \end{math}

\subsection{Chaîne directe}

Rajoutons maintenant un correcteur PID et calculons la fonction de transfert de la chaîne directe.


\subsection{Extraction de sortie temporelle}

\end{document}
